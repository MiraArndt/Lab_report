\chapter{Einleitung}

Durch die fortschreitende Miniaturisierung technischer Bauteile spielen die 
physikalischen Effekte an Oberflächen und Grenzflächen dünner Schichtsysteme 
eine immer größere Rolle.\newline
In der Vergangenheit wurden im Bereich der Spinelektronik bereits viele relevante Entdeckungen gemacht, deren 
Anwendungen heutzutage in der Industrie weit verbreitet sind.
Als Beispiel kann der gigantische Magnetwiderstand 
aufgeführt werden, zu dessen vielen Anwendungen Spin-Ventile gehören,
welche flächendeckend in Leseköpfen von Computerfestplatten genutzt werden \cite{PhysRevLett.61.2472,PhysRevB.39.4828,daughton2000gmr,jogschies2015recent}. 
Ein verwandter Effekt ist der magnetische Tunnelwiderstand, 
der genutzt wird um magnetische Tunnelkontakte
zu erzeugen \cite{JULLIERE1975225}. Bereits hier wurde die Relevanz von Fe/MgO/Fe-Systemen
wegen ihres großen Tunnelwiderstandes deutlich \cite{parkin2004giant}.\newline
Gegenstand aktueller Forschung sind molekülbasierte Spintronics \cite{cinchetti2017activating},
bei denen neue Spineffekte an der Grenzfläche zwischen anorganischen und molekularen Schichten untersucht werden.
Die Verwendung von Molekülen bringt dabei mehrere Vorteile mit sich, wie zum Beispiel eine 
lange Spin-Relaxationszeit, selbstständige Anordnung und die Möglichkeit, chemische und physikalische Eigenschaften für 
verschiedene Aufgaben anzupassen \cite{cinchetti2009determination,sun2018progress}. 
Zu den Anwendungen gehören beispielsweise organische licht-emittierende Dioden und Photovoltaic Zellen \cite{koch2007organic}.\newline
Der Einfluss von MgO als dielektrische Zwischenschicht in metallisch/organischen Hybridsystemen wurde  2017 bereits von
Hollerer et al. am Beispiel von Ag(100)/Pentacen bzw. Ag(100)/MgO/Pentacen Grenzschichten diskutiert. Durch das Einführen der 
MgO-Zwischenschicht konnte für den Ladungstransfer zwischen Metall und Molekül ein Übergang von fraktionellem hin zu 
ganzzahligem Ladungstransfer beobachtet werden, welcher auf den einsetzenden Tunneleffekt zurückzuführen ist.
Ein ähnlicher Effekt ist bei der Verwendung von Eisen als Substrat denkbar, 
wobei die im ferromagnetischen Eisen vorliegende Spin-Polarisation auch einen einheitlichen,
spinabhängigen Ladungstransfer zur Folge hätte. Dies könnte widerum Anwendung in zukünftigen spinelektronischen Bauteilen  finden.\newline
Die Herstellung dieser Systeme setzt ein Verständnis über das Wachstum der MgO-Schichten voraus.
Zu diesem Zweck wird in dieser Arbeit das MgO-Wachstum auf Fe(100) und Fe(100)-p(1\,x\,1)O
mit oberflächensensitiven Methoden charakterisiert und verglichen.
Mit den gewonnenen Referenzdaten ist es möglich das Wachstum zu reproduzieren,
auch wenn nicht alle hier verwendeten Methoden zur Verfügung stehen.
Eine Untersuchung der Wachstumsschichtdicke erfolgt duch 
reflektierte Elektronenbeugung und Augerelektronenspektroskopie. 
Letztere Methode wird auch genutzt um eine Elementanalyse vorzunehmen und so die Stöchiometrie zu prüfen.
Zudem wird die Oberflächenstruktur durch Elektronenbeugung charakterisiert.

%Die Stöchiometrie wird mittels einer Elementanalyse geprüft und die 
%Oberflächenstruktur durch Elektronenbeugung charakterisiert.
%

%
%Als Beispiel kann der magnetische Tunnelkontakt (magnetic tunnel junction, "MTJ")
%genannt werden, bei dem die Bedeutung von Fe/MgO/Fe-Systemen
%wegen ihres großen Tunnelwiderstandes gezeigt wurde \cite{parkin2004giant}.\newline
%Durch die fortschreitende Miniaturisierung technischer Bauteile spielen die 
%physikalischen Effekte an Oberflächen und Grenzflächen dünner Schichtsysteme 
%eine immer größer Rolle.\newline
%Dabei wurden im Bereich der Spinelektronik bereits viele relevante Entdeckungen gemacht, deren 
%Anwendungen heutzutage in der Industrie weit verbreitet sind.
%Als Beispiel können die beiden spinabhängigen Effekte gigantische Magnetoresistenz (giant magnetoresistance) 
%und magnetische Tunnelwiderstände (tunnel Magnetoresistance) aufgeführt werden, zu deren vielen Anwendungen zum Beispiel Spin-Ventile gehören,
%welche verbreitet bei der Datenspeicherung genutzt werden \cite{daughton2000gmr,jogschies2015recent}. 
%
%Fertig

%So beschreibt ein Modell von Barraud et al. \cite{barraud2010unravelling} etwa, dass die Spin-Polarisation von ferromagnetischen Materialien 
%an der Grenzfläche zu molekularen Schichten durch die Grenzflächenhybridisierung beeinflussst werden kann.


%Bei solch einem molekularem System wurde bereits von Hollerer et al. \cite{hollerer2017charge}
%die Relevanz von MgO als dielektrische Zwischenschicht entdeckt. In solch einem System ist der übergang von einem 
%fraktionellen Ladungstransfer (durch Hybridisierung) ohne MgO-Zwischenschicht zu einem ganzzahligen Ladungstransfer (durch Elektronentunneln)
%mit MgO-Zwischenschicht zu beobachten.
%



%Eisen spielt als Ferromagnet bei Raumtemperatur eine entscheidende Rolle bei 
%spinabhängigigen Effekten \cite{parkin2004giant,yuasa2004giant}.
%So könnte Eisen genutzt werden, um durch Spinpolarisation das Tunneln 
%der Elektronen durch die MgO Zwischenlage und somit den ganzzahligen Ladungstransfer bei einem Fe/MgO/Pentacene-System 
%in Abhängigkeit des Spins zu manipulieren.
%Die starken magnetischen Eigenschaften an der Oberfläche von reinem Eisen 
%können durch Passivierung und das Erzeugen einer (1\,x\,1)-Überstruktur noch verstärkt werden \cite{tange2010electronic}.
%
%Fertig
%Eisen spielt als Ferromagnet bei Raumtemperatur eine entscheidende Rolle bei 
%spinabhängigigen Effekten \cite{parkin2004giant,yuasa2004giant}. Die starken magnetischen Eigenschaften an der Oberfläche von reinem Eisen 
%können durch Passivierung und das Erzeugen einer (1\,x\,1)-Überstruktur noch verstärkt werden \cite{tange2010electronic}.\newline
%Dielektrische Zwischenschichten an metallischen/molekularen Grenzflächen können zur Entkopplung oder als Tunnelbarriere dienen. 
%So wurden bereits die bedeutenden Einflüsse einer dielektrischen MgO-Zwischenschicht 
%im Dreilagensystem Ag/MgO/Pentacene von Hollerer et al. ausgearbeitet \cite{hollerer2017charge}.
%
%Fertig
