%%%%%%%%%%%%%%%%%%%%%%%%%%%%%%%%%%%%%%%%%%%%%%%%%%%%%%%%%%%%%%%%%%%%%%%%%%%%%%%%
%%%%%%%%%%%%%%%%%%   Vorlage für eine Abschlussarbeit   %%%%%%%%%%%%%%%%%%%%%%%%
%%%%%%%%%%%%%%%%%%%%%%%%%%%%%%%%%%%%%%%%%%%%%%%%%%%%%%%%%%%%%%%%%%%%%%%%%%%%%%%%

% Erstellt von Maximilian Nöthe, <maximilian.noethe@tu-dortmund.de>
% ausgelegt für lualatex und Biblatex mit biber

% Kompilieren mit 
% latexmk --lualatex --output-directory=build thesis.tex
% oder einfach mit:
% make

\documentclass[
% tucolor,       % remove for less green,
  BCOR=12mm,     % 12mm binding corrections, adjust to fit your binding
  parskip=half,  % new paragraphs start with half line vertical space
  open=any,      % chapters start on both odd and even pages
  cleardoublepage=plain,  % no header/footer on blank pages
]{tudothesis}
\renewcommand*{\chapterheadstartvskip}{\vspace*{.5\baselineskip}}
%\chapterheadstartvskip
% Warning, if another latex run is needed
\usepackage[aux]{rerunfilecheck}

% just list chapters and sections in the toc, not subsections or smaller
\setcounter{tocdepth}{1}

%------------------------------------------------------------------------------
%------------------------------ Fonts, Unicode, Language ----------------------
%------------------------------------------------------------------------------
\usepackage{fontspec}
\defaultfontfeatures{Ligatures=TeX}  % -- becomes en-dash etc.

% german language
\usepackage{polyglossia}
\setdefaultlanguage{english}

% for english abstract and english titles in the toc
%\setotherlanguages{english}

% intelligent quotation marks, language and nesting sensitive
\usepackage[autostyle]{csquotes}

% microtypographical features, makes the text look nicer on the small scale
\usepackage{microtype}

%------------------------------------------------------------------------------
%------------------------ Math Packages and settings --------------------------
%------------------------------------------------------------------------------

\usepackage{amsmath}
\usepackage{amssymb}
\usepackage{mathtools}

% Enable Unicode-Math and follow the ISO-Standards for typesetting math
\usepackage[
  math-style=ISO,
  bold-style=ISO,
  sans-style=italic,
  nabla=upright,
  partial=upright,
]{unicode-math}
\setmathfont{Latin Modern Math}

% nice, small fracs for the text with \sfrac{}{}
\usepackage{xfrac}  


%------------------------------------------------------------------------------
%---------------------------- Numbers and Units -------------------------------
%------------------------------------------------------------------------------

\usepackage[
  locale=DE,
  separate-uncertainty=true,
  per-mode=symbol-or-fraction,
]{siunitx}
\sisetup{math-micro=\text{µ},text-micro=µ,output-decimal-marker={.}}

%------------------------------------------------------------------------------
%-------------------------------- tables  -------------------------------------
%------------------------------------------------------------------------------

\usepackage{booktabs}       % \toprule, \midrule, \bottomrule, etc

%------------------------------------------------------------------------------
%-------------------------------- graphics -------------------------------------
%------------------------------------------------------------------------------

\usepackage{graphicx}
% currently broken
% \usepackage{grffile}

\usepackage{sidecap}

% allow figures to be placed in the running text by default:
\usepackage{scrhack}
\usepackage{float}
\floatplacement{figure}{htbp}
\floatplacement{table}{htbp}

% keep figures and tables in the section
\usepackage[section, below]{placeins}


%------------------------------------------------------------------------------
%---------------------- customize list environments ---------------------------
%------------------------------------------------------------------------------

\usepackage{enumitem}
\usepackage{pdfpages}
%------------------------------------------------------------------------------
%------------------------------ Bibliographie ---------------------------------
%------------------------------------------------------------------------------

\usepackage[
  backend=biber,   % use modern biber backend
  autolang=hyphen, % load hyphenation rules for if language of bibentry is not
  sorting=none               % german, has to be loaded with \setotherlanguages
                   % in the references.bib use langid={en} for english sources
]{biblatex}
\addbibresource{references.bib}  % the bib file to use
\DefineBibliographyStrings{german}{andothers = {{et\,al\adddot}}}  % replace u.a. with et al.


% Last packages, do not change order or insert new packages after these ones
\usepackage[pdfusetitle, unicode, linkbordercolor=tugreen]{hyperref}
\usepackage{bookmark}
\usepackage[shortcuts]{extdash}


%==============================================================================
\usepackage{upgreek}

%------------------------------------------------------------------------------
%-------------------------    Angaben zur Arbeit   ----------------------------
%------------------------------------------------------------------------------

%\author{David Gutnikov \and Mira Sophie Arndt}
\author{%
  David Gutnikov\\%
  \href{mailto:david.gutnikov@tu-dortmund.de}{david.gutnikov@tu-dortmund.de} \and
  Mira Arndt\\
  \href{mailto:mira.arndt@tu-dortmund.de}{mira.arndt@tu-dortmund.de}%
}
\title{Photoemission study of the Fe(100)-p(1x1)O/ZnTPP and Cu(100)/ZnTPP interfaces}
\date{2022}
%\birthplace{Unna}
\chair{AG Cinchetti}
\division{Department of Physics}
%\thesisclass{Lab Report}
%\submissiondate{04. 11. 2022}
%\firstcorrector{Prof.~Dr.~Mirko Cinchetti}
%\secondcorrector{Prof.~Dr.~Markus Betz}
%\thesisbetreuer{David~Janas}
%\secondthesisbetreuer{Dr.~Giovanni~Zamborlini}


% tu logo on top of the titlepage
% \titlehead{\includegraphics[height=1.5cm]{logos/tu-logo.pdf}}

\begin{document}
\frontmatter
\maketitle

% Gutachterseite
%\makecorrectorpage

% hier beginnt der Vorspann, nummeriert in römischen Zahlen
%\input{content/00_abstract.tex}
\tableofcontents

\mainmatter
% Hier beginnt der Inhalt mit Seite 1 in arabischen Ziffern
%\chapter{Grundlagen des Systems}
\section{Fe(100), Fe(100)-p(1\,x\,1)O und MgO}

Eisen ist bei Raumtemperatur ferromagnetisch und kristalliert in einer kubisch-raumzentrierten Kristallstruktur 
mit einer Gitterkonstante von $2,87\,\si{\angstrom}$. 
Die Bezeichnung Fe(100) bezieht sich auf die Miller-Indizes ($h\,k\,l$) und gibt die Orientierung der Kristalloberfläche im Raum an.

Bei der Passivierung mit Sauerstoff setzen sich die Sauerstoffionen auf die Muldenplätze der Fe(100)-Struktur und bilden eine (1\,x\,1)-Überstruktur \cite{jona1987re}.
Die chemische Stabilität der Oberfläche erhöht sich auf diese Weise und die magnetischen Eigenschaften werden gleichzeitig verstärkt \cite{tange2010electronic}.

\begin{SCfigure}
    \centering
    \includegraphics[width=0.45\linewidth]{Plots/FeO_Struktur.png}
    \caption{Schematische Darstellung der Fe(100)-p(1\,x\,1)O-Oberfläche in der Seitenansicht mit Angabe der durch die Passivierung verursachten Deformationen der Fe(100) Oberfläche \cite{jona1987re}.}
    \label{fig:FeO}
\end{SCfigure}

Magnesiumoxid kristallisiert in einer NaCl-Struktur mit einer Gitterkonstante von $4,21\,\si{\angstrom}$.
Eine einzelne Monolage im Festkörper ist demnach $2,105\,\si{\angstrom}$ groß, mit vernachlässigbaren Abweichungen
an der Grenzfläche zu Fe(100)- und Fe(100)-p(1\,x\,1)O-Oberflächen \cite{meyerheim2001geometrical}.




\section{Arten des Schichtwachstums}
\label{sec:Wachstum}

Idealerweise verläuft das Wachstum des Adsorbats lagenweise auf dem Substrat (Frank-van-der-Merwe-Wachstum), sodass Atome erst dann 
eine neue Monolage ("ML") bilden, wenn die vorherige ML vollständig belegt ist.
Ist die Grenzflächenenergie des wachsenden Films zum Substrat und zum Vakuum größer als 
die Energie des Substrats zum Vakuum, so kommt es zu Inselwachstum (Volmer-Weber-Wachstum).
Da es energetisch günstiger ist, möglichst wenig Fläche des Substrats zu bedecken, bildet sich ein Film, der weder einkristallin 
noch von homogener Dicke ist.
Wird dem System Energie in Form von Wärme zugefügt, kann diese Energiedifferenz überwunden werden und bevorzugt ein lagenweises Wachstum 
stattfinden. Die Mischform aus anfänglich lagenweisem Wachstum und anschließendem Inselwachstum wird Stranski-Krastanov-Wachstum genannt \cite{fauster}.

\begin{SCfigure}
    \centering
    \includegraphics[width=0.45\linewidth]{Plots/Fe_MgO_Struktur.png}
    \caption{Anordnung einer Lage MgO auf Fe(100) in der Draufsicht. Die Sauerstoffionen liegen auf den Eisenatomen, während die Magnesiumionen 
    nach der NaCl-Struktur von MgO zwischen den Sauerstoffplätzen liegen.
    Die Richtungsangaben beziehen sich auf das Fe-System \cite{meyerheim2001geometrical}.}
    \label{fig:Fe_MgO}
\end{SCfigure}

Wenn sowohl Substrat als auch Adsorbat einkristallin sind 
und die wachsende Schicht eine wohldefinierte Beziehung zum Substrat hat, spricht man von Epitaxie.
Beim Wachstum von MgO auf Fe(100) und Fe(100)-p(1\,x\,1)O liegt die [100]-Richtung des Magnesiums in [110]-Richtung 
des Eisens, sodass die Sauerstoffionen auf den Eisenatomen liegen. Es entsteht eine Struktur wie schematisch in Abbildung \ref{fig:Fe_MgO} zu sehen, bei der 
die Gitterkonstante von MgO um $3,7\,\si{\percent}$ vom Atomabstand $\sqrt{2}\cdot 2,87\,\si{\angstrom}=4,06\,\si{\angstrom}$ von Eisen in [110]-Richtung abweicht. 
MgO kann bis zu einer Schichtdicke von 6 ML epitaktisch wachsen \cite{klaua2001growth}, erst danach bilden sich Versetzungen in Folge der 
Fehlanpassung der Gitterkonstanten \cite{dynna1996low}.

Beim Aufdampfen von Mg in einer Sauerstoffatmosphäre kann die Stöchiometrie direkt kontrolliert werden. 
Tekiel et al. \cite{tekiel2013reactive} fanden ein optimiertes Verhältnis der Mg-Aufdampfrate $r$ zum Sauerstoffdruck $p$ von 

\begin{equation}
    \dfrac{r}{p}=(0,15\pm 0,05)\cdot 10^8 \,\si{\angstrom\per\milli\bar\per\minute}.
    \label{eq:V1}
\end{equation}

Auf diese Weise können MgO-Filme mit hoher Stöchiometrie und kristalliner Struktur produziert werden.
%\chapter{Methoden}

\section{Quarzkristall-Mikrowaage}
Eine Möglichkeit, die Masse  eines aufgedampften Materials
zu bestimmen, ist die Änderung der Eigenfrequenz 
einer angeregten Quarzplatte (Quartz Crystal Microbalance, "QCM") zu untersuchen.
Durch die Vergrößerung der schwingenden Masse $\symup{\Delta}m$ ändert sich die Frequenz $\symup{\Delta}f$ nach der Sauerbrey Gleichung \cite{sauerbrey1959verwendung},
was auf den linearen Zusammenhang

\begin{equation}
        \symup{\Delta}f = -\symup{C} \cdot \symup{\Delta}m
\end{equation}

führt. Der Faktor $\symup{C}$ bezeichnet eine Kalibrierungskonstante. Aus der aufgedampften Masse lässt sich über die Dichte des Materials 
die gewachsene Schichtdicke und damit die Aufdampfrate in $\si{\angstrom\per\minute}$ bestimmen.
Diese gibt einen Anhaltspunkt für die spätere Aufdampfrate auf der untersuchten Probe, 
weicht aber um einen konstanten Faktor von der tatsächlichen Rate ab. 
Eine Ursache dafür sind unterschiedliche Adsorptionsraten des Quarzkristalls und der untersuchten Probe.




\section{LEED und IV-LEED}

Zur Strukturbestimmung in Festkörpern eignen sich allgemein Beugungsmethoden mit Teilchen, deren Wellenlänge in der Größenordnung der 
zu untersuchenden Struktur liegt. Bei der Beugung niederenergetischer Elektronen (Low Energy Electron Diffraction, "LEED") werden Elektronen 
genutzt, deren Energien im Bereich $E=50$-$200$\,eV liegen, sodass sie zum Einen passend für eine atomare Auflösung sind und zum Anderen durch ihre geringe mittlere freie Weglänge 
eine oberflächensensitive Methode bilden. 

Dabei trifft ein Elektronenstrahl aus einer Elektronenkanone senkrecht auf die Probe, welcher dort gebeugt und anschließend 
zurück auf einen  Fluoreszenz-basierten Leuchtschirm trifft. 
Inelastisch gestreute Elektronen werden durch Gitter mit passender Gegenspannung herausgefiltert.
Hinter den Gittern werden die elastisch gestreuten Elektronen zum Leuchtschirm hin beschleunigt.
Die dort entstehenden Beugungsreflexe werden mit einer Kamera aufgenommen.

\begin{SCfigure}
        \centering
        \includegraphics[width=0.5\linewidth]{Plots/LEED_Schema.png}
        \caption{Schematische Darstellung einer LEED-Optik. In grün ist die Bahn der elastisch gestreuten Elektronen zu sehen,
        welche unter dem Winkel $\alpha$ an der Probe gebeugt werden und auf den gekrümmten Leuchtschirm mit Radius $R$ treffen.
        Am Gitter liegt eine Abbremsspannung $U_{\symup{B}}$ an, um inelastisch gestreute Elektronen auszufiltern \cite{fauster}.}
        \label{fig:LEED_Schema}
\end{SCfigure}

Die Beugungsintensität $I$ hängt vom Formfaktor $F$ und vom Gitterfaktor $G$ ab
\begin{equation}
        I=|F|^2\cdot |G|^2.
        \label{eq:LEED}
\end{equation}

Es gilt nur dann $G\neq 0$, wenn die Laue-Bedingung $\symup{\Delta}\vec{k}_{||}=\vec{g}_{hk}$ für den Impulsübertrag in senkrechter Richtung erfüllt ist.
$\vec{g}_{hk}$ bezeichnet hier einen diskreten reziproken Gittervektor.
Es treten also nur bestimmte Beugungsreflexe auf, die durch die Bezeichnung ($hk$) identifiziert werden.
Die Beugungsreflexe geben ein Bild des reziproken Raums der Oberfläche wieder
und sind charakteristisch für die Oberflächenstruktur der Probe.
Ist die Ordnung der Oberfläche zum Beispiel durch Fremdatome gestört, treten weitere Streueffekte auf und
die Reflexe verlieren an Schärfe.

Nach Gleichung \ref{eq:LEED} wirken sich unterschiedliche Strukturen direkt auf die Reflex-Intensität aus, 
weshalb  diese zur Charakterisierung eines Materials verwendet werden kann.
Dafür werden so genannte IV-Kurven genutzt, die die Intensitätsverläufe einzelner Reflexe in Abhängigkeit der Energie 
wiedergeben.
\newpage

\section{MEED}
Die Beugung mittelenergetischer Elektronen (Medium-Energy Electron Diffraction, "MEED") kann in einem Reflektionsaufbau \ref{fig:MEEDS} 
genutzt werden, um die Anzahl der Monolagen während des Wachstums zu bestimmen.

\begin{figure}[H]
        \centering
        \includegraphics[width=\linewidth]{Plots/MEED_3.pdf}
        \caption{Schematische Darstellung der MEED-Methode im Reflexionsaufbau. Die Elektronenkanone befindet sich gegenüber des
                Leuchtschirms, der Elektronenstrahl kann so in einem flachen Winkel $\Theta$ auf die Probe treffen.
                Dort werden die Elektronen wie bei der LEED-Methode elastisch gestreut und bilden ein Beugungsbild am Leuchtschirm.
                Befindet sich der Aufdampfer \cite{FOCUS} wie abgebildet vor der Probe, kann das Beugungsbild 
                während des Wachstums beobachtet werden. Die Intensität der Beugungsreflexe oszilliert mit der Schichtdicke.
                Mit freundlicher Genehmigung von David Janas.}
        \label{fig:MEEDS}
\end{figure}

Die Intensität der Beugungsreflexe oszilliert dabei in Abhängigkeit der Struktur der Oberfläche.
Die beste Ordnung und demnach die höchste Intensität besteht bei einer abgeschlossenen vollen Monolage.
Bei einem lagenweisen Wachstum kann nun die Bildung einer neuen Lage zunächst durch einen Abfall der Intensität beobachtet werden, 
auf den ein erneuter Anstieg folgt, bis die neue Monolage vollständig ist. Anhand der Anzahl der Intensitätsmaxima lässt sich so die
Anzahl der Monolagen beim Wachstum dünner Schichten bestimmen \cite{neave1983dynamics}. 




\newpage
\section{Augerelektronenspektroskopie}
\label{sec:Auger}
Eine oberflächensensitive Methode zur quantitativen Elementanalyse einer Probe ist die Augerelektronenspektroskopie (Auger Elektron Spectroscopy, "AES").
Mit Hilfe der mittleren freien Weglänge von Elektronen in den entsprechenden Systemen lässt sich aus den Peak-Verhältnissen eines Auger-Spektrums außerdem die Schichtdicke bestimmen.

Wird in einem Atom ein Elektron aus einer inneren Schale z.B. durch Elektronenstrahlung herausgelöst,
wird das entstehende Loch durch ein Elektron einer äußeren Schale aufgefüllt. 
Wenn die dabei frei werdende Energie an ein weiteres Elektron übertragen wird,
kann dieses den Festkörper verlassen und wird Augerelektron genannt.
Dessen Energie ist dabei abhängig von den beteiligten Energieniveaus des Atoms, und somit elementspezifisch.
Wird die Anzahl der detektierten Elektronen in Abhängigkeit ihrer Energie aufgetragen, bilden sich bei den entsprechenden Energien 
charakteristische Peaks, wie beispielhaft in Abbildung \ref{fig:Auger-BSP} gezeigt. Die Intensität wird dabei in differentieller Form dargestellt,
um die Peaks vor dem Untergund inelastisch rückgestreuter Elektronen und Sekundärelektronen hervorzuheben.

\begin{figure}[H]
        \centering
        \includegraphics[width=0.7\linewidth]{Plots/plotAuger2_Layout.pdf}
        \caption{Augerspektrum einer dünnen MgO-Schicht auf Fe(100). Die Elementzugehörigkeiten der gemessenen Peaks sind entsprechend gekennzeichnet.}
        \label{fig:Auger-BSP}
\end{figure}


Um unterschiedliche elementspezifische Faktoren, wie die Wahrscheinlichkeit für einen Augerzerfall,
bei den relativen Peak-Verhältnissen einbeziehen zu können,
werden relative Sensitivitäten $S_{\symup{i}}$ des Elements i definiert, welche empirisch bestimmt werden \cite{davis-1978}.
Die für das zu untersuchende System wichtigen Sensitivitäten von Fe, O und Mg sind in Tabelle \ref{tab:AugerS} aufgetragen.
Die Elementkonzentration $c_{\symup{i}}$ kann so mit der Gleichung 

\begin{equation}
        c_{\symup{i}}=\dfrac{{I_{\symup{i}}}/{S_{\symup{i}}}}{\sum_{\symup{j}}{I_{\symup{j}}}/{S_{\symup{j}}}}
        \label{eq:AugerK}
\end{equation}

bestimmt werden, wobei $I_{\symup{i}}$ die Peak-Intensität bezeichnet. Die Summe läuft über alle vorhandenen Elemente j \cite{fauster}.

\begin{table}[H]
        \centering
        \begin{tabular}{S | S}
          \toprule
          {$\symup{Element}$} & {$S_{\symup{i}}$}\\
          \midrule
          $\symup{Fe \:(651\,eV)}$ & {$0,2$}\\
          $\symup{O \:(503\,eV)}$ & {$0,5$} \\
          $\symup{Mg \:(1174\,eV)}$ & {$0,1$}\\
          \bottomrule
        \end{tabular}
        \caption{Relative Sensitivitäten $S_{\symup{i}}$ der untersuchten Elemente Fe, O und Mg \cite{davis-1978}.}
        \label{tab:AugerS}
\end{table}
Um die Schichtdicke aus Augerspektren zu bestimmen, ist es wichtig, die mittleren freien Weglängen $\lambda$ von Elektronen bei den Energien zu kennen, 
bei denen die charakteristischen Peaks auftreten. In der Literatur lassen sich verschiedene mittlere freie Weglängen finden,
wovon einige in Tabelle \ref{tab:AugerM} aufgeführt sind.

\begin{table}
        \centering
        \begin{tabular}{c| c c c c}
          \toprule
          Energie & Gries \cite{gries1996universal} & Tanuma et al. \cite{tanuma1994calculations}& Akermann et al. \cite{akkerman1996inelastic}& Seah et al. \cite{seah1979quantitative}\\
          \midrule
          {$651\,\si{\eV}$} & {$14,62\,\si{\angstrom}$} & {$16,35\,\si{\angstrom}$} & {$15,48\,\si{\angstrom}$} & {$18,9\,\si{\angstrom}$}\\
          {$503\,\si{\eV}$} & {$12,70\,\si{\angstrom}$} & {$13,65\,\si{\angstrom}$} & {$12,69\,\si{\angstrom}$} & {$16,7\,\si{\angstrom}$}\\
          {$1174\,\si{\eV}$} & {$22,55\,\si{\angstrom}$}& {$25,34\,\si{\angstrom}$}& {$24,72\,\si{\angstrom}$}& {$23,5\,\si{\angstrom}$}\\
          \bottomrule
        \end{tabular}
        \caption{Mittlere freie Weglängen bei den relevanten Energien in MgO \cite{Wachstum}.}
        \label{tab:AugerM}
      \end{table}


\begin{equation}
        \dfrac{I_{\symup{A}}}{I_{\symup{S}}}=\dfrac{S_{\symup{A}}\cdot \left( 1-\symup{exp}\left( -\dfrac{d}{0,74\cdot \lambda_{\symup{A}}} \right) \right) }  {S_{\symup{S}}\cdot \symup{exp} \left( -\dfrac{d}{0,74 \cdot\lambda_{\symup{S}}} \right) }
        \label{eq:Auger-V}
\end{equation} 

Gleichung \ref{eq:Auger-V} stellt einen Zusammenhang zwischen dem Augerpeak-Verhältnis des Adsorbats $I_{\symup{A}}$ zum Substrat $I_{\symup{S}}$ und der Schichtdicke $d$
des Absorbats her \cite{Wachstum}. 
\newpage


\section{Experimentelles Setup}

\begin{SCfigure}
        \centering
        \includegraphics[width=0.6\linewidth]{Plots/BILD_PREP.pdf}
        \caption{Gezeigt ist die verwendete Präperationskammer. Beschriftet sind die Komponenten des Reflektions-MEED Aufbaus.}
        \label{fig:BILD_PREP}
\end{SCfigure}

Das Wachstum wurde in einer Ultra-Hoch Vakuum (UHV) Kammer durchgeführt, die auf das Wachstum und die Analyse dünner Schichtsysteme ausgelegt ist.
Der Basisdruck in der Kammer liegt in der Größenordnung von $p=6\cdot 10^{-11}\,\si{\milli\bar}$.
Zur ständigen Druckmessung wird dabei ein Ionisations-Vakuummeter verwendet.
In die Kammer können über Leckventile $\symup{Ar}^+$-Ionen sowie $\symup{O_2}$-Moleküle 
eingelassen werden. 

Die $\symup{Ar}^+$-Ionen werden dabei zum Reinigen der Probe mittels ioneninduzierter Zerstäubung (sputtering) verwendet.
Sie können durch eine angelegte Spannung unter einem Winkel auf die Probe beschleunigt werden und tragen durch Stöße Oberflächenatome ab.
Anschließendes Heizen der Probe mit einem Heizwiderstand sorgt für die Desorption weiterer Verunreinigungen sowie eine Neuordnung der 
Oberfläche durch zusätzliche thermische Energie.
 Die Temperatur wird mit einem Thermoelement gemessen, welches sich leicht entfernt von der Probe befindet, weshalb 
eine kleine Differenz zur tatsächlichen Temperatur der Probe vorliegt.

Die Probe ist in der Kammer durch einen Manipulator frei translatierbar und zudem um $360\,\si{\degree}$ um die z-Achse rotierbar.
Am Manipulator ist außerdem eine QCM befestigt, die in der Höhe und  Tiefe versetzt zur Probenhalterung angebracht ist.

Das Magnesium wird aus Pellets mit $99,99\,\si{\percent}$ Reinheit aus einem Molybdän-Tiegel aufgedampft.
Der Aufdampfer steht dabei in einem Winkel von $90\,\si{\degree}$ zum LEED-Schirm und zur Elektronenkanone des AES, was MEED-Messungen während des Aufdampfens ermöglicht.
Für unterschiedliche Winkel zwischen Probe und LEED-Schirm stellen sich verschiedene Beugungsordnungen ein, welche in Abbildung \ref{fig:MEED-Bilder} zu sehen sind.



\begin{figure}[H]
    \centering
    \subfloat[][]{\includegraphics[width=0.28\linewidth]{Plots/MEED_103_degree.png}}%
    \qquad
    \subfloat[][]{\includegraphics[width=0.28\linewidth]{Plots/MEED_109_degree.png}}%
    \qquad
    \subfloat[][]{\includegraphics[width=0.28\linewidth]{Plots/MEED_113_degree.png}}%
    \caption{Bilder des LEED-Schirms mit MEED-Einstellungen für verschiedene Winkel zwischen Probe und Schirm. 
            (a) zeigt einen Winkel von $\Theta=13\,\si{\degree}$, (b) einen Winkel von $\Theta=7\,\si{\degree}$ und (c) einen Winkel von $\Theta=3\,\si{\degree}$ zwischen Probe und Schirm.}%
    \label{fig:MEED-Bilder}
  \end{figure}



%\chapter{Preparationskammer}

Das Wachstum wurde in einer Ultra-Hoch Vakuum (UHV) Kammer durchgeführt, die auf das Wachstum und die Analyse dünner Schichtsysteme ausgelegt ist.
Der Basisdruck in der Kammer liegt in der Größenordnung von $p=1\cdot 10^{-10}\,\si{\milli\bar}$.
Zur ständigen Druckmessung wird dabei ein Ionisations-Vakuummeter (ion gauge) verwendet.
In die Kammer können über leak Ventile $\symup{Ar}^+$-Ionen sowie $\symup{O_2}$-Moleküle 
eingelassen werden. 

Die $\symup{Ar}^+$-Ionen werden dabei zum Reinigen der Probe bei der Ioneninduzierten Zerstäubung (sputtering) verwendet.
Diese können durch eine angelegte Spannung unter einem Winkel auf die Probe beschleunigt werden und tragen durch Stöße Oberflächenatome ab.
Durch anschlißendes Heizen der Probe desorbieren weitere Verunreinigungen und durch die zusätzliche thermische Energie ordnet sich die 
Oberfläche neu.

Die Probe ist in der Kammer durch einen Manipulator frei translatierbar und außerdem um $360\,\si{\degree}$ um die z-Achse rotierbar.
Am Manipulator ist außerdem eine QCM befestigt, die in der Höhe und  Tiefe versetzt zur Probenhalterung angebracht ist.

Der MG-Aufdampfer steht im $90\,\si{\degree}$ Winkel zum LEED Schrim und zur Elektronenquelle des AES, was MEED Messungen während des Aufdampfens ermöglicht.
Für unterschiedliche Winkel zwischen Probe und LEED Schirm stellen sich verschiedene Beugungsordnungen ein, welche in Abbildung \ref{fig:MEED-Bilder} zu sehen sind.



\begin{figure}[H]
    \centering
    \subfloat[][]{\includegraphics[width=0.28\linewidth]{Plots/MEED_103_degree.png}}%
    \qquad
    \subfloat[][]{\includegraphics[width=0.28\linewidth]{Plots/MEED_109_degree.png}}%
    \qquad
    \subfloat[][]{\includegraphics[width=0.28\linewidth]{Plots/MEED_113_degree.png}}%
    \caption{Bilder des LEED Schirms mit MEED Einstellungen für verschiedene Winkel zwischen Probe und Schirm. 
            (a) zeigt einen Winkel von $13\,\si{\degree}$, (b) einen Winkel von $7\,\si{\degree}$ und (c) einen Winkel von $3\,\si{\degree}$ zwischen Probe und Schirm.}%
    \label{fig:MEED-Bilder}
  \end{figure}


%\chapter{Ergebnisse}
Die verwendete Fe(100)-Oberfläche wurde gefertigt, 
indem ein dünner Eisenfilm (ca. $200\,\si{\nano\meter}$\,-\,$400\,\si{\nano\meter}$) auf einen MgO(100)-Kristall aufgedampft wurde.

Um eine saubere Fe(100)-Oberfläche zu erhalten,
wurde die Probe zunächst durch mehrere Sputter- und Heizdurchgänge
gereinigt. Bei je einem Durchgang wurde bei einer Spannung von $U=1\,\si{\kilo\volt}$ und 
einem Argondruck von $p=1\cdot 10^{-5}\,\si{\milli\bar}$ für 
$15\,\si{\minute}$ gesputtert
und für $5\,\si{\minute}$ bei einer Temperatur von $T=600\,\si{\celsius}$ geheizt. Mittels Auger-Spektren und LEED-Bildern 
wurde die Reinheit und Ordnung der Oberfläche überprüft. 
Solch ein Auger-Spektrum und ein LEED-Bild von der sauberen Fe(100) Oberfläche sind in den Abbildungen \ref{fig:Auger1} 
(oben) und  \ref{fig:LEED} (a) zu sehen.



Auf der gereinigten Probe wurde nun das MgO-Wachstum einzelner Monolagen untersucht.
Die Mg-Aufdampfrate wurde mittels der QCM bestimmt 
und damit anschließend bei passendem Sauerstoffdruck nach Zusammenhang \ref{eq:V1} 
eine MEED-Messung des Wachstums von MgO bei einer Probentemperatur von $T=170\,\si{\celsius}$
für $22\,\si{\minute}$ auf dem sauberen Eisen durchgeführt. 
Eine höhere Temperatur erleichtert zwar das lagenweise Wachstum, doch 
nach Tekiel et al. sinkt die kristalline Ordnung bei Temperaturen über $T=160\,\si{\celsius}$ \cite{tekiel2013reactive}. 
Wegen der Distanz des Thermoelements zur Probe wurde hier eine etwas höhere Temperatur von $T=170\,\si{\celsius}$ gewählt.

Um den Prozess mit Auger-Spektren, LEED-Bildern und IV-Kurven zu untersuchen, wurde das gleiche Wachstum noch einmal 
sukzessive durchgeführt. Dabei wurde die Probe je einmal vollständig mit diesen Methoden vermessen und anschließend 
bei den gleichen Einstellungen wie oben Magnesium nacheinander für je $t=116\,\si{\second}$, $t=250\,\si{\second}$, $t=597\,\si{\second}$ 
und $t=251\,\si{\second}$ in Sauerstoffatmosphäre aufgedampft.

Für den Vergleich mit dem Wachstum von MgO auf passivertem Eisen wurde auch dieses System vermessen.
Zur Passiverung wurde die saubere Eisenprobe zunächst für $5\,\si{\minute}$ bei einer Temperatur von $T=550\,\si{\celsius}$ 
und einer Sauerstoffatmosphäre mit 
$p=1,3\cdot 10^{-7}\,\si{\milli\bar}$ ausgesetzt, was $30\,\symup{L}$ entspricht. 
Nach \cite{picone2016controlling} bildet sich auf diese Weise eine Fe(100)-p(1\,x\,1)O Struktur.
\newpage
\section{Schichtdicken-Charakterisierung mittels MEED und AES}

In Abbildung \ref{fig:QCM1} ist die aufgedampfte Mg-Schichtdicke in Abhängigkeit der Zeit aufgetragen.
Im Bereich von $t=400\,\si{\second}$ bis $t=600\,\si{\second}$ wurde der Aufdampfer mit einem Flux von $130\,\si{\nano\ampere}$ betrieben, 
welcher als finale Einstellung auch beim MgO-Wachstum verwendet wurde.

\begin{SCfigure}
  \centering
  \includegraphics[scale=1, width=0.5\linewidth]{Plots/QCM_Layout.pdf}
  \caption{Aufgedampfte Mg-Schichtdicke auf der QCM vor und nach dem Öffnen des Shutters.
          Aus der Änderung der Schichtdicke mit der Zeit ergibt sich eine Aufdampfrate von $r=0,34\,\si{\angstrom\per\minute}$.}
  \label{fig:QCM1}
\end{SCfigure}

Aus der dadurch resultierenden Aufdampfrate von $r=0,34\,\si{\angstrom\per\minute}$ und dem verwendeten 
Sauerstoffdruck $p=3\cdot 10^{-8}\,\si{\milli\bar}$ ergibt sich ein Verhältnis von

\begin{equation*}
  \frac{r}{p}=0,11\cdot 10^8 \,\si{\angstrom\per\milli\bar\per\minute},
\end{equation*}

welches im Bereich $r/p=(0,15\pm 0,05)\cdot 10^8 \,\si{\angstrom\per\milli\bar\per\minute}$ (siehe Abschnitt \ref{sec:Wachstum}) liegt.

\begin{figure}[H]
  \centering
  \includegraphics[scale=1, width=0.7\linewidth]{Plots/MEED-Original-2_Layout.pdf}
  \caption{MEED-Spektrum des MgO-Wachstums auf Fe(100). Aufgedampft wurde mit einer Rate von $r=0,34\,\si{\angstrom\per\minute}$ 
          bei einem Sauerstoffdruck von $p=3\cdot 10^{-8}\,\si{\milli\bar}$.}
  \label{fig:MEED1}
\end{figure}

Der in Abbildung \ref{fig:MEED1} gezeigte MEED-Verlauf zeigt die Maxima der ersten zwei vollen Monolagen nach $106\,\si{\second}$ und weiteren $281\,\si{\second}$.
Aufgrund des starken Untergrunds lässt sich das Maxima der dritten Monolage nicht so eindeutig und  das der vierten Monolage gar nicht zuordnen, 
doch der relative Verlauf und die Ergebnisse der späteren Auswertung der Augerpeak Verhältnisse lässt darauf schließen, dass sich
die dritte Monolage nach weiteren $352\,\si{\second}$ und die vierte Monolage sich nach weiteren $345\,\si{\second}$ bildet. 

Auffällig ist hier, dass die Wachstumszeit für eine volle Monolage nicht konstant ist, sondern sich einem Grenzwert von ca. $350\,\si{\second}$ annähert.

Aus der Mg-Aufdampfrate der QCM lässt sich eine Vorhersage für die MgO-Wachstumsrate erstellen. Dazu wird angenommen, dass 
bei einem MgO-Kristall die Hälfte der Gitterplätze mit Sauerstoffatomen belegt ist, wodurch sich die MgO-Wachstumsrate gegenüber der Mg-Aufdampfrate
verdoppeln würde. Jedoch ist auch die Dichte von MgO um den Faktor $\rho_{\symup{MgO}}/\rho_{\symup{Mg}}=2,06$ größer als die von Mg, weshalb
sich die Wachstumsrate durch die dichtere Anordnung verringert. Insgesamt ergibt sich so $r_{\symup{Wachstum,\, QCM}}=0,97\cdot r=0,33\,\si{\angstrom\per\minute}$.
Zwischen der so ermittelten und der gemessenen Wachstumsrate $r_{\symup{Wachstum,\, MEED}}=0,36\,\si{\angstrom\per\minute}$ von MgO (angenommen wurde eine Schichtdicke von 2,105$\si{\angstrom}$) 
besteht also ein Faktor 

\begin{equation*}
  \frac{r_{\symup{Wachstum,\, MEED}}}{r_{\symup{Wachstum,\, QCM}}}=1,09.
\end{equation*}

Um die variierende Wachstumszeit genauer zu untersuchen, wird eine 
weitere MEED-Messung des Wachstums von MgO auf Fe(100) und Fe(100)-p(1\,x\,1)O 
zum Vergleich der beiden Systeme betrachtet.
Die Aufdampfrate für diese Messung wurde 
mittels der QCM zu $r=0,45\,\si{\angstrom\per\minute}$ bestimmt (Abbildung \ref{fig:QCM2}).

\begin{SCfigure}
  \centering
  \includegraphics[scale=1, width=0.5\linewidth]{Plots/QCM_2021_07_21_Layout.pdf}
  \caption{Aufgedampfte Mg-Schichtdicke auf der QCM vor und nach dem Öffnen des Shutters.
  Aus der Änderung der Schichtdicke mit der Zeit ergibt sich eine Aufdampfrate von $r=0,45\,\si{\angstrom\per\minute}$.}
  \label{fig:QCM2}
\end{SCfigure}



\begin{figure}[H]
  \centering
  \includegraphics[scale=1, width=0.7\linewidth]{Plots/MEED_2021_07_21_FeO_u_Fe_Layout.pdf}
  \caption{MEED-Messungen des MgO-Wachstums auf Fe(100) (orange) und Fe(100)-p(1\,x\,1)O (blau). Die Messungen wurden um einen linear verlaufenden Untergrund korrigiert.
          Aufgedampft wurde mit einer Rate von $r=0,45\,\si{\angstrom\per\minute}$ 
          bei einem Sauerstoffdruck von $p=4\cdot 10^{-8}\,\si{\milli\bar}$.}
  \label{fig:MEED2}
\end{figure}

Die Oszillationen im MEED-Verlauf \ref{fig:MEED2} sind bei dieser Messung deutlich zu erkennen und können durchgängig 
den vollen Monolagen des Wachstums zugeordnet werden. Beim passivierten Eisen ist die Wachstumszeit nahezu konstant bei $258\,\si{\second}$ für eine volle Monolage.
Auch hier stellt sich der Faktor von

%Die Oszillationen im MEED Verlauf \ref{fig:MEED2} sind bei dieser Messung deutlich zu erkennen und können durchgängig eindeutig 
%den vollen Monolagen des Wachstums zugeordnet werden. Hierbei ist die Wachstumszeit nahezu konstant bei $258\,\si{\second}$ für eine volle Monolage.
%Auch hier stellt sich der Faktor von
%
\begin{equation*}
  \frac{r_{\symup{Wachstum,\, MEED}}}{r_{\symup{Wachstum,\, QCM}}}=1,09
\end{equation*} 

zwischen der Wachstumsrate $r_{\symup{Wachstum,\, QCM}}=0,97\cdot r=0,44\,\si{\angstrom\per\minute}$ und 
der bestimmten Wachstumsrate $r_{\symup{Wachstum,\, MEED}}=0,48\,\si{\angstrom\per\minute}$ ein.

Es ist wieder zu beobachten, dass die ersten zwei ML auf dem reinen Eisen schneller wachsen als die Nachfolgenden.
Später stellt sich die gleiche Wachstumszeit für eine ML ein wie beim passivierten Eisen.
Dies lässt sich auf die hohe Reaktivität der reinen Eisenoberfläche zurückführen.
Während also bei dem MgO Wachstum auf reinem Eisen der Sauerstoff besonders schnell auf der Eisenoberfläche adsorbiert, 
so ist das passivierte Eisen schon mit Sauerstoff gesättigt und das Wachstum von Beginn an gleichbleibend schnell.

Die Schichtdickenbestimmung mit AES-Spektren wurde bei MgO-Wachstum auf sauberem Fe(100) durchgeführt.
Die aufgenommenen Spektren des sukzessiven Wachstums sind in Abbildung \ref{fig:Auger1} dargestellt.

\begin{SCfigure}
  \centering
  \includegraphics[scale=1, width=0.7\linewidth]{Plots/Auger_Vergleich_Layout.pdf}
  \caption{Auger-Spektren im Energiebereich $E=200-1250\, \si{\eV}$ nach unterschiedlicher Aufdampfdauer. 
          Die Bestimmung der Anzahl der ML erfolgt später durch \ref{fig:Auger-Verhältnisse2}. Durch einen Savitzky-Golay Filter 2.\nobreakspace Ordnung mit 9 Punkten wurden die
          Auger-Spektren geglättet.}
  \label{fig:Auger1}
\end{SCfigure}



\begin{figure}[H]
  \centering
  \includegraphics[width=\linewidth]{Plots/Auger-Verhältnisse-Schichtdicke_2_Layout.pdf}
  \caption{Augerpeak-Verhältnisse in Abhängigkeit der Schichtdicke errechnet mit Gleichung \ref{eq:Auger-V}, für die verschiedenen
           mittleren freien Weglängen aus \ref{tab:AugerS}. Dabei geben die blauen Kurven das O/Fe Verhältnis und die schwarzen Kurven das Mg/Fe Verhältnis an.
           Die Kreuze kennzeichnen die gemessenen Verhältnisse nach ausgewählten Aufdampfdauern.}
  \label{fig:Auger-Verhältnisse1}
\end{figure}

Die Auswertung der Verhältnisse der Peaks für verschiedene mittlere freie Weglängen nach Gleichung \ref{eq:Auger-V} ist in Abbildung \ref{fig:Auger-Verhältnisse1} zu sehen.
Dabei stimmen die Schichtdicken der Mg/Fe und der O/Fe-Verhältnisse mit den mittleren freien Weglängen von Seah et al. \cite{seah1979quantitative} am besten überein.
Diese sind noch einmal in Abbildung \ref{fig:Auger-Verhältnisse2} dargestellt.

\begin{SCfigure}
  \centering
  \includegraphics[width=0.7\linewidth]{Plots/Auger-Verhältnisse-Schichtdicke_Layout.pdf}
  \caption{Augerpeak-Verhältnisse in Abhängigkeit der Schichtdicke errechnet mit Gleichung \ref{eq:Auger-V} und den mittleren freien Weglängen von Seah et al. \cite{seah1979quantitative}, je einmal für das Mg/Fe- und das O/Fe-Verhältnis.
          Die Kreuze kennzeichnen die gemessenen Verhältnisse nach ausgewählten Aufdampfdauern.}
  \label{fig:Auger-Verhältnisse2}
\end{SCfigure}


Auf diese Weise lassen sich den Wachstumszeiten jeweils präzise MgO-Schichtdicken zuordnen.
Da es sich um unterschiedliche Wachstumsdurchläufe handelt,
lassen sich die Ergebnisse nur bedingt auf die MEED-Messung \ref{fig:MEED1} übertragen, 
weichen jedoch auch dort nur um maximal $0,5$\,ML von der bestimmten Schichtdicke ab.


\section{Elementanalyse mittels AES}


Aus den aufgenommenen Auger-Spektren \ref{fig:Auger1} lässt sich durch Gleichung \ref{eq:AugerK} die Konzentration der Elemente an der Oberfläche bestimmen,
siehe Tabelle \ref{tab:AugerT}.


\begin{table}
  \centering
  \begin{tabular}{S| S S S S S}
    \toprule
    {$\symup{Element}$} & {$0\,\symup{ML}$} & {$1\,\symup{ML}$} & {$1,6\,\symup{ML}$} & {$3,5\,\symup{ML}$} & {$4,3\,\symup{ML}$}\\
    \midrule
    $\symup{Fe}$ & {$100\,\si{\percent}$} & {$77\,\si{\percent}$} & {$66\,\si{\percent}$} & {$44\,\si{\percent}$} & {$37\,\si{\percent}$}\\
    $\symup{O}$ & {$0\,\si{\percent}$} & {$11\,\si{\percent}$} & {$20\,\si{\percent}$} & {$32\,\si{\percent}$} & {$36\,\si{\percent}$}\\
    $\symup{Mg}$ & {$0\,\si{\percent}$}& {$12\,\si{\percent}$}& {$14\,\si{\percent}$}& {$24\,\si{\percent}$}& {$27\,\si{\percent}$}\\
    \bottomrule
  \end{tabular}
  \caption{Konzentration der vorhandenen Elemente an der Oberfläche nach sukzessivem Wachstum von MgO auf Fe(100).}
  \label{tab:AugerT}
\end{table}

Während zu Beginn die Konzentration des Magnesiums noch gleich der von Sauerstoff ist, stellt sich nach dem zweiten Wachstum ein 
Sauerstoffüberschuss ein. Es scheint also sinnvoll, den Sauerstoffdruck nach der ersten Monolage zu senken, was den Ergebnissen aus \cite{tekiel2013reactive} widerspricht.
Tekiel et al. fanden ein besseres Mg/O-Verhältnis bei niedrigeren Sauerstoffdrücken zu Beginn des Wachstums während der 
ersten Monolage, mit darauf folgenden höheren Sauerstoffdrücken.

Es ist jedoch wichtig anzumerken, dass der Mg-Peak durch die geringe Sensitivität $S=0,1$ bei sehr geringer Konzentration an
der Oberfläche nur schwer vom Untergundrauschen der Augerspektren zu unterscheiden ist. 

\begin{SCfigure}
  \centering
  \includegraphics[scale=1, width=0.69\linewidth]{Plots/Auger_Vergleich__FeO_Layout.pdf}
  \caption{Auger-Spektren im Energiebereich $E=200-1200\, \si{\eV}$ bei unterschiedlichen Schichtdicken von MgO auf Fe-p(1\,x\,1)O. 
          Durch einen Savitzky-Golay Filter 2. Ordnung mit 9 Punkten wurden die
          Auger-Spektren geglättet.}
  \label{fig:Auger2}
\end{SCfigure}



\begin{table}[H]
  \centering
  \begin{tabular}{S| S S S}
    \toprule
    {$\symup{Element}$} & {$0\,\symup{ML}$} & {$1,1\,\symup{ML}$} & {$2,5\,\symup{ML}$}\\
    \midrule
    $\symup{Fe}$ & {$89\,\si{\percent}$} & {$71\,\si{\percent}$} & {$61\,\si{\percent}$}\\
    $\symup{O}$ & {$11\,\si{\percent}$} & {$19\,\si{\percent}$} & {$21\,\si{\percent}$}\\
    $\symup{Mg}$ & {$0\,\si{\percent}$}& {$10\,\si{\percent}$}& {$18\,\si{\percent}$}\\
    \bottomrule
  \end{tabular}
  \caption{Konzentration der vorhandenen Elemente an der Oberfläche nach sukzessivem Wachstum von MgO auf Fe(100)-p(1\,x\,1)O.}
  \label{tab:AugerT2}
\end{table}

Für MgO auf dem passivierten Eisen sind die Auger-Spektren in Abbildung \ref{fig:Auger2} dargestellt und die errechneten Konzentrationen 
in Tabelle \ref{tab:AugerT2} aufgeführt. Während die Mg-Konzentration vergleichbar ist mit der des Wachstum auf reinem Fe(100),
bleibt die Sauerstoffkonzentration beim passivierten Eisen auch nach dem Wachstum weniger Monolagen erhöht. 
Dies zeigt widerum, dass die reine Eisenprobe nicht
sofort passiviert wird, wenn für das Wachstum eine Sauerstoffatmopshäre in der Größenordnung $p=3\cdot 10^{-8}\,\si{\milli\bar}$ eingestellt wird. 
Weiterführend dazu fanden Cattoni et al. \cite{PhysRevB.80.104437} mittels Röntgenphotoelektronenspektroskopie-Messungen heraus, dass keine Fe-O Interaktionen an der Fe(100)/MgO-Grenzfläche 
stattfinden. Die physikalischen Eigenschaften der reinen Eisenoberfläche bleiben im Falle eines 
Fe(100)/MgO-Systems im Vergleich zu einem Fe(100)-p(1\,x\,1)O/MgO-System also erhalten. Zudem wurde bei diesen Schichtsystemen 
eine erhöhte Austauschaufspaltung für die nicht-passivierte Oberfläche beobachtet.


\section{LEED- und IV-LEED-Messungen}


\begin{figure}[H]
  \centering
  \subfloat[][]{\includegraphics[width=0.25\linewidth]{Plots/LEED_neu/clean_Fe.png}}%
  \qquad
  \subfloat[][]{\includegraphics[width=0.25\linewidth]{Plots/LEED_neu/1ML_Fe.png}}%
  \qquad
  \subfloat[][]{\includegraphics[width=0.25\linewidth]{Plots/LEED_neu/1_6ML_Fe.png}}%
  \qquad
  \subfloat[][]{\includegraphics[width=0.25\linewidth]{Plots/LEED_neu/2ML_Fe.png}}%
  \qquad
  \subfloat[][]{\includegraphics[width=0.25\linewidth]{Plots/LEED_neu/3_5ML_Fe.png}}%
  \qquad
  \subfloat[][]{\includegraphics[width=0.25\linewidth]{Plots/LEED_neu/4_3ML_Fe.png}}%
  \caption{LEED-Bilder nach unterschiedlichen Aufdampfdauern von MgO auf Fe(100) bei einer Energie von $E=96\,\si{\eV}$. (a) zeigt sauberes Fe(100),
          (b) eine $1$\,ML Schicht, (c) eine $1,6$\,ML Schicht, (d) eine $2$\,ML Schicht, (e) eine $3,5$\,ML Schicht und (f) eine $4,3$\,ML Schicht.}%
  \label{fig:LEED}
\end{figure}


In Abbildung \ref{fig:LEED} sind auf den LEED-Bildern bei den gleichen Energien $E=96\,\si{\eV}$
die Reflexe bei verschiedenen Schichtdicken von MgO auf reinem Fe(100) zu sehen. 
Die LEED-Bilder können genutzt werden, um eine starke oder schwächere Ordnung der Oberfläche zu 
erkennen und so zum Beispiel zu beurteilen, ob das Wachstum auf der gesamten Probe gleichmäßig verlaufen ist.  
Die unterschiedliche Schärfe der Reflexe ist hier wahrscheinlich auf die unterschiedlichen zugrundeliegenden Wachstumsbedingungen zurückzuführen.



Bei den LEED-Bildern von MgO auf Fe(100)-p(1\,x\,1)O \ref{fig:LEED1}, welche alle im gleichen Wachstumsprozess bei ähnlichen Einstellungen aufgenommen wurden,
ist eine nachlassende Schärfe der Reflexe bei teilweise gefüllten Monolagen zu sehen. 
Bei einer Schichtdicke von $1,1$\,ML liegt eine fast gefüllte ML vor, sodass die Schärfe im Vergleich zur sauberen
Probe etwas nachlässt. Die Reflexe der $2,5$\,ML Schicht sind noch diffuser, 
was sich durch die halbzahlige ML und die dadurch entstehende geringe Ordnung erklären lässt.


%Bei einer Schichtdicke von $1,1$\,ML liegt eine fast gefüllte ML vor,
%die Schärfe lässt im Vergleich zur sauberen Probe etwas nach, 
%bei $2,5$\,ML lässt sie weiter nach, was sich durch die halbzahlige ML und die dadurch entstehende geringe Ordnung erklären lässt.


\begin{figure}[H]
  \centering
  \subfloat[][]{\includegraphics[width=0.25\linewidth]{Plots/LEED_neu/clean_FeO.png}}%
  \qquad
  \subfloat[][]{\includegraphics[width=0.25\linewidth]{Plots/LEED_neu/1_1ML_FeO.png}}%
  \qquad
  \subfloat[][]{\includegraphics[width=0.25\linewidth]{Plots/LEED_neu/2_5ML_FeO.png}}%

  \caption{LEED-Bilder nach unterschiedlichen Aufdampfdauern von MgO auf Fe(100)-p(1\,x\,1)O bei einer Energie von $E=65\,\si{\eV}$. (a) zeigt die saubere Fe(100)-p(1\,x\,1)O Probe,
          (b) eine $1,1$\,ML Schicht und (c) eine $2,5$\,ML Schicht.}%
  \label{fig:LEED1}
\end{figure}

In Abbildung \ref{fig:LEED2} sind zwei LEED-Bilder von $1,6$\,ML MgO auf reinem Fe(100) zu sehen.
Dabei wurde \ref{fig:LEED2}(a) vor einem schnellen Aufheizen auf $T=600\,\si{\celsius}$ aufgenommen und \ref{fig:LEED2}(b)
danach. Die klareren Reflexe nach dem Aufheizen sprechen für eine bessere Ordnung des Systems.
Dies könnte auf das Entstehen größerer MgO-Domänen hindeuten oder auf das stellenweise Wachstum von Inseln, 
welche durch das Aufheizen aufgebrochen werden.
%was 
%darauf hindeutet, dass Inselstrukturen aufgelöst wurden \ref{sec:Wachstum}. Die  unvollständige Ordnung durch die unbeendete Monolage konnte also 
%durch das Heizen doch noch verbessert werden.

\begin{figure}[H]
  \centering
  \subfloat[][]{\includegraphics[width=0.25\linewidth]{Plots/LEED_neu/1_6_ML_Fe_96eV_vor_annealen.png}}%
  \qquad
  \subfloat[][]{\includegraphics[width=0.25\linewidth]{Plots/LEED_neu/1_6_ML_Fe_96eV_nach_annealen.png}}%
  \caption{LEED-Bilder einer $1,6$\,ML Schicht MgO auf Fe(100) bei einer Energie von $E=96\,\si{\eV}$. (a) zeigt die Probe vor dem Aufheizen,
          (b) zeigt die Probe nach dem Aufheizen.}%
  \label{fig:LEED2}
\end{figure}




\begin{figure}[H]
  \centering
  \begin{subfigure}{0.47\textwidth}
    \includegraphics[width=\textwidth]{Plots/IV-clean_Layout.pdf}
    \caption{}
  \end{subfigure}
  \begin{subfigure}{0.47\textwidth}
    \vspace{-0.2cm}
    \includegraphics[width=\textwidth]{Plots/fe.png}
    \caption{}
  \end{subfigure}
  \begin{subfigure}{0.47\textwidth}
    \includegraphics[width=\textwidth]{Plots/IV-FeO_Layout.pdf}
    \caption{}
  \end{subfigure}
  \begin{subfigure}{0.47\textwidth}
    \vspace{-0.2cm}
    \includegraphics[width=\textwidth]{Plots/feO.png}
    \caption{}
  \end{subfigure}
  \begin{subfigure}{0.47\textwidth}
%    \vspace{-0.2cm}
    \includegraphics[width=\textwidth]{Plots/IV-MgO_Layout.pdf}
    \caption{}
  \end{subfigure}
  \begin{subfigure}{0.47\textwidth}
    \vspace{0.6cm}
    \includegraphics[width=\textwidth]{Plots/mgo.png}
    \caption{}
  \end{subfigure}
  \caption{Aufgenommene IV-Kurven links (a) von sauberem Fe(100), (c) sauberem Fe(100)-p(1\,x\,1)O und (e) MgO. 
          Der (00)-Reflex vom reinen Eisen wurde unter einem $15\,\si{\degree}$ Winkel aufgenommen, der vom passivierten Eisen bei $10\,\si{\degree}$.
          Durch einen Savitzky-Golay Filter 2. Ordnung mit 21 Punkten wurden die
          IV-Kurven geglättet.
          Rechts zum Vergleich Literaturdaten aus dem IV Data Repository \cite{IV}, welche den Daten aus \cite{PhysRevB.16.5271,Legg_1977,JONA1987667,URANO1983109} entsprechen.}%
  \label{fig:IV1}
\end{figure}


Für das reine Eisen, das mit Sauerstoff passivierte Eisen und einen dicken MgO-Film (mehr als 10\,ML)
wurden IV-Kurven aufgenommen. In Abbildung \ref{fig:IV1} ist ein Vergleich der gemessenen Kurven (links) für die 
drei Systeme mit der Literatur (rechts) gezeigt.
Bei dem (10)- und dem (11)-Reflex des passivierten Eisens sind Abweichungen bei den Peakhöhen zu erkennen.
Abgesehen davon liegt eine gute Übereinstimmung der restlichen Reflexe vor.
Die abweichenden Ergebnisse für das passivierte Eisen konnten mehrfach reproduziert werden.


Durch das sukzessive Wachstum ist es möglich, einen zeitlichen Verlauf der IV-Kurven aufzunehmen und Referenzdaten
für eine unterschiedliche Anzahl weniger Monolagen zu erhalten. In Abbildung \ref{fig:IV2} ist eine Zusammenstellung 
dieser IV-Kurven für den (10)-, den (11)- und den (20)-Reflex auf Fe(100) (links) und Fe(100)-p(1\,x\,1)O (rechts) zu sehen.


Die Markierungen zeigen Charakteristika verschiedener Schichtdicken.
Beim MgO-Wachstum auf reinem Eisen ist beim (10)-Reflex in Abbildung \ref{fig:IV2}(a) ein Peak 
bei $240$\,eV bei 1\,ML zu sehen, der bei den anderen vermessenen Schichtdicken nicht auftritt.\newline
Der (11)-Reflex in Abbildung \ref{fig:IV2}(c) zeigt einen Rückgang des $170$\,eV-Peaks vom sauberen Eisen Fe(100), welcher bei 1\,ML kleiner und ab $1,6$\,ML gar nicht mehr sichtbar ist.
Der $140$\,eV-Peak vom MgO-Verlauf ist bereits deutlich ab $1,6$\,ML sichtbar und bleibt bei höheren Schichtdicken konstant.\newline
Ebenso ist der $165$\,eV-Peak des (20)-Reflexes in Abbildung \ref{fig:IV2}(e) von MgO bereits ab $1,6$\,ML zu beobachten, wobei dieser 
bei $1,6$\,ML und $3,5$\,ML noch vom $190$\,eV-Peak überlagert wird. Letzterer nimmt mit zunehmender Schichtdicke an Intensität ab.

Beim MgO-Wachstum auf passiviertem Eisen zeigt der (10)-Reflex in Abbildung \ref{fig:IV2}(b) bereits ab einer Schichtdicke von 1,1\,ML den 
$165$\,eV-Peak von MgO.\newline 
Gleich verhält es sich mit dem $140$\,eV-Peak beim (11)-Reflex in Abbildung \ref{fig:IV2}(d).
Im Vergleich zum Wachstum auf reinem Eisen (siehe Abbildung \ref{fig:IV2}(c)) ist auffällig, dass dieser dort bei einer Schichtdicke von 1\,ML 
noch nicht sichtbar ist.\newline
Ähnlich wie beim (20)-Reflex des Wachstums auf reinem Eisen ist beim (20)-Reflex auf passiviertem Eisen in Abbildung \ref{fig:IV2}(f)
eine Überlagerung zweier Peaks zu sehen. Der $165$\,eV-Peak von MgO überlagert sich ab einer Schichtdicke von 1,1\,ML
mit dem $190$\,eV-Peak, wobei letzterer auch wieder mit steigender Schichtdicke abnimmt.
Wie beim $140$\,eV-Peak beim (11)-Reflex, ist der $190$\,eV-Peak hier beim Wachstum auf reinem Eisen (siehe Abbildung \ref{fig:IV2}(e)) bei einer Schichtdicke von
1\,ML noch nicht sichtbar.

Die aufgeführten Merkmale der IV-Kurven können genutzt werden, um verschiedene Schichtdicken der 
Systeme Fe(100)/MgO und Fe(100)-p(1\,x\,1)O/MgO zu identifizieren.


\begin{figure}[H]
   \centering
   \subfloat[][]{\includegraphics[width=0.47\linewidth]{Plots/IV-10_Layout.pdf}}%
   \qquad
   \subfloat[][]{\includegraphics[width=0.47\linewidth]{Plots/IV-10_FeO-Layout.pdf}}
   \qquad
   \subfloat[][]{\includegraphics[width=0.47\linewidth]{Plots/IV-11_Layout.pdf}}
   \qquad
   \subfloat[][]{\includegraphics[width=0.47\linewidth]{Plots/IV-11_FeO-Layout.pdf}}
   \qquad
   \subfloat[][]{\includegraphics[width=0.47\linewidth]{Plots/IV-20_Layout.pdf}}%
   \qquad
   \subfloat[][]{\includegraphics[width=0.47\linewidth]{Plots/IV-20_FeO-Layout.pdf}}
   \caption{IV-Kurven aufgenommen bei unterschiedlicher Schichtdicke von MgO auf Fe(100) (links) und MgO auf Fe(100)-p(1\,x\,1)O (rechts).
            (a) und (b) zeigen jeweils den (10)-Reflex, (c) und (d) den (11)-Reflex und (e) und (f) den (20)-Reflex.
            Durch einen Savitzky-Golay Filter 2. Ordnung mit 21 Punkten wurden die
          IV-Kurven geglättet.}%
   \label{fig:IV2}
\end{figure}

%\chapter{Zusammenfassung und Ausblick}

In dieser Arbeit wurde das Wachstum von MgO sowohl auf Fe(100)- als auch auf Fe(100)-p(1\,x\,1)O-Oberflächen
mit verschiedenen Methoden untersucht. 
Es wurden Daten aufgenommen, die das Wachstum charakterisieren und als mögliche Referenz dienen,
dünne Fe(100)/MgO- und Fe(100)-p(1\,x\,1)O/MgO-Schichtsysteme präzise herzustellen.
%
%Es wurden Referenzdaten dieser Methoden präsentiert, die das Wachstum charakterisieren und  
%somit eine Möglichkeit geben, dünne Fe(100)/MgO- und Fe(100)-p(1\,x\,1)O/MgO-Schichtsysteme präzise zu reproduzieren.

Die Ergebnisse zur Bestimmung der Schichtdicke mittels AES- und MEED-Messungen 
komplementieren sich gegenseitig.
Es hat sich herausgestellt, dass die ersten beiden Monolagen auf Fe(100) schneller 
wachsen und sich anschließend eine Wachstumsrate einstellt, die der konstanten Rate vom Fe(100)-p(1\,x\,1)O/MgO-System entspricht.\newline
Außerdem konnte die für solch ein System charakteristische Abweichung zwischen der ermittelten Wachstumsrate und der 
vorhergesagten QCM-Wachstumsrate bestimmt werden. 

Durch LEED-Aufnahmen vor und nach dem Erhitzen konnte ein positiver Effekt 
des Aufheizens auf die Oberflächenordnung des Fe(100)/MgO-Systems nachgewiesen werden.
Ein Aufbrechen von Inseln bei $1,6$\,ML als Ursache widerspricht den Ergebnissen von Klaua et al. \cite{klaua2001growth} und 
außerdem den Ergebnissen der durchgeführten MEED-Messungen. Als komplementäre Methode zur Untersuchung dieses Effekts 
könnte ein Rastertunnelmikroskop genutzt werden.\newline
Weiterhin wurden charakteristische Merkmale in den IV-Kurven beider Systeme 
für unterschiedliche Schichtdicken herausgearbeitet.

Die Ergebnisse für das geeignete $r/p$-Verhältnis von Tekiel et al. \cite{tekiel2013reactive} konnten nicht bestätigt werden,
denn wie die Elementanalyse mit AES gezeigt hat, entsteht beim Wachstum mehrerer Monolagen ein Sauerstoffüberschuss.\newline
Auffällig ist jedoch, dass die Schichtdickenanalyse mittels AES trotzdem passende Ergebnisse 
sowohl für das Mg/Fe- als auch für das O/Fe-Verhältnis erzeugt hat. 
Weitere Messungen bei verändertem Sauerstoffdruck wären nötig, um dieses Phänomen weiter zu untersuchen. 


Ein hier nicht betrachteter Aspekt ist der gezielte Einbau von Defekten mittels verändertem r/p-Verhältnis 
in der MgO-Schicht. So kann die Austrittsarbeit verändert werden und folglich die Eigenschaften molekularer 
Hybridsysteme gezielt manipuliert werden \cite{greiner2012universal}. 
Notwendig dazu wäre eine Vermessung der Austrittsarbeit zum Beispiel mittels Photoelektronenspektroskopie.
%Ein weiterer Aspekt des Wachstums, der hier nicht betrachtet wurde, ist der gezielte Einbau von Defekten 
%in der MgO-Schicht, um eine Änderung der Austrittsarbeit herbeizuführen.
%Solch eine Veränderung der Austrittsarbeit kann genutzt werden, um die Eigenschaften von molekularen Spinterface-Schichtsystemen 
%gezielt zu manipulieren \cite{greiner2012universal}. \newline
%Gegenstand zukünftiger Untersuchungen könnte also eine Charakterisierung des MgO-Wachstums mit verändertem 
%$r/p$-Verhältnis sein, um gezielt Defekte einzubauen. Damit verbunden wäre eine Vermessung der Austrittsarbeit 
%des Systems zum Beispiel mittels Photoelektronenspektroskopie. 


%Der Einbau solcher Defekte kann durch die hier verwendete Wachstumsmethode durch eine Veränderung des $r/p$-Verhältnises
%vorgenommen werden. Der Versuch das MgO-Wachstum bei Sauerstoffmangel zu charakterisieren ist im Rahmen dieser Arbeit fehlgeschlagen.






%Ein weiterer Effekt, der eine genauere Betrachtung wert ist, ist der Anstieg der MEED-Intensität \ref{fig:MEED1}
%direkt nach dem Öffnen des Shutters. Erwartet wird zunächst eine niedrigere Ordnung der Oberfläche durch das Auftreffen 
%der ersten Fremdatome, welche eine Senkung der Beugungsintensität zur Folge haben sollten.
%Eine solche Senkung ist auch bei den weiteren MEED-Messungen \ref{fig:MEED2} zu erkennen.
%MEED-Messungen unter verschiedenen Bedinungen, vor allem der Variation des Winkels und der vermessenen Beugungsreflexe
%könnten Aufschluss über diesen unerwarteten Effekt bringen.



\chapter{Introduction}
The flexibility of porphyrins in organic devices such as solar cells [6] and OLEDs [7], as well their biological importance [9] are some of the reasons why porphyrines are subject of in-depth study [8].
Especially the adsorbtion on metal surfaces and the placing of a metall ion in the main cavity of the molecule contribute greatly to the tunability of the porphyrin properties [10].
The understanding of the fundamental mechanics involved in the interaction between adlayer and substrate as well as the physical and chemical properties of the interface work as a basis for creating new devices.

Tetraphenylporphyrin consists of a tetrapyrrolic macrocycle, which lies almost flat on the substrate and therefore interacts the most with the substrate.
The four phenyl groups are situated on the border of the macrocycle and are almost perpendicular to the macrocycle in the unperturbed state.
The metall ion located in the middle of the pyrrole ring stays in the plane of the ring, meaning that it is accessible from both sides.

Here we wanted to reproduce some of the results for FeO/ZnTPP found in literatur [2] and build upon them to study the interface of Cu(100)/ZnTPP in a similar manner.
The unpassivated Cu surface is naturally expected to undergo greater change in its electronic structure after adsorption of the ZnTPP molecules.

As previous works show, a direct deposition of ZnTPP on a clean Fe(001) surface produced no LEED pattern [4], which is said to be caused by island formation disrupting the flat crystallographic structure.
Passivation of this highly reactive Fe substrate was shown to prevent strong coupling to the porphyrins, so that eventhough the pyrrole ring lies flat on the substrate the main electronic features like the HOMO and LUMO are conserved [5].
The ZnTPP then also form a (5\times5)-reconstruction, showcasing the enhanced molecule diffusivity compared to the clean Fe surface [4].

\chapter{Methods}

The multichamber ultra high vacuum (UHV) system with a base pressure in the range of $\qty{e-10}{mbar}$ used in these experiments consists of a loadlock to get the samples inside the setup, a preparation chamber where the samples are cleaned and the molecule deposition takes place and lastly the analysis chamber where the measurements are taken.

The first step in the preparation of the Fe(001) surface was cleaning by sputtering the sample with $\mathrm{Ag}^+$-ions at a kinetic energy of \qty{1}{keV} and an emission current of \qty{6}{mA} for 15 minutes and then annealing for 7 minutes after reaching \qty{600}{°C} to rearange the surface to be atomically flat.
Afterwards, the sample was passivated with the base pressure rising to \qty{1.0e-7}{mbar} for 5 minutes at \qty{550}{°C}, exposing the surface to about 23 Langmuirs of oxygen.
Next, a flash annealing procedure upto \qty{600}{°C} took place to limit the heat treatment to the surface area and to remove the extra oxygen atoms, leaving behind a Fe monoxide layer.
The oxygen atoms lie in the hollow sites situated on the Fe(001) surface.
The Cu(100) surface underwent a similar cleaning procedure.
Here the kinetic energy for sputtering was set to \qty{2}{keV} for a duration of 70 minutes.
Zinc tetraphenylporphyrin (ZnTPP) was deposited on the newly formed Fe(001)p-(1\times1)O and the Cu(100) surfaces via Knudsen effusion.
For this the effusion cell containing the ZnTPP was heated to \qty{265}{°C} and then opened for 12 minutes aiming perpendicularly at the sample.

Low Energy Electron Diffraction (LEED) is a basic technique to examine the crystallographic structure of a sample by scattering electrons elastically on the sample surface and observing the spots, generated on a flourescent screen, corresponding to the reciprocal lattice vectors due to the Laue equations.
This method is used to inspect the coverage of the monoxide layer and to check the quality of the super-lattice consisting of porphyrins.
To confirm the chemical composition of the prepared sample Auger Electron Spectroscopy (AES) is used.
The electronic bandstructure is probed via Ultraviolet Photoemission Spectroscopy (UPS), where the obtained spectra relay information about the valance bands near the fermi level.
The angle resolved photoemission spectroscopy (ARPES) used to record the two dimensional momentum maps is realized with the KREIOS system.
The impuls component parallel to the surface needed for the maps is computed using the emission angle of the extracted electrons, which in turn is captured by the hemispherical analyzer.
To excite the electrons, for the photoemission to happen, high harmonic generated (HHG) light with a photon energy of \qty{29,8}{eV} was used.
For both surfaces the analyzer was set to a wider range of kinetic energies to study some of the main features and to a narrow range to obtain the work function and information pertaining to the surface dipole.
All measurements were done with an energy resolution of $\Delta E = \qty{50}{meV}$.

\chapter{Results}

Here, the results of the data analysis should be reported in a concise way. The data should be
properly analyzed and compared with the literature, highlighting similarities, as well as
differences, with other works. There is not need for too many figures, the idea is trying to
summarize the main data that you want to show in few images and “condensate” the
information. For the report between 5 to 10 figures are ok. 
\chapter{Conclusions}

Through LEED and AES measurments a successfull preperation of both studied substrates could be validated.
Even though there could not be found a commensurate super-lattice for the 1ML coverage of ZnTPP on Cu(100), the sharp spots clearly show an ordered assembly of the molecules.
The momentum map at fermi energy as well as the main bands in the bandstructure of the Fe(100)-\textit{p}(1\times1)O sample are comparible to literature, only the features around the $\overline{\Upgamma}$ point at the fermi level could not be reproduced.
For further studies of the molecular features in k-space, it would be recommended to perform measurements at a synchrotron or collect data with higher statistics, because no molecular features where visible in the obtained momentum maps.
Nonetheless the main molecular peaks where evident in the angle integrated spectra comparable to literature. [tesi]
In addition to that, we could confirm a higher interaction for the reactive Cu(100) surface with the ZnTPP molecules in comparison to the passivated Fe(100) surface, through an observed reduced workfunction.







%\input{content/01_einleitung.tex}
%\input{content/02_struktur.tex}
%\input{content/03_grundlagen.tex}
%\input{content/04_figs_tabs.tex}

\appendix
% Hier beginnt der Anhang, nummeriert in lateinischen Buchstaben
%\input{content/a_anhang.tex}
\backmatter
\printbibliography

\end{document}
