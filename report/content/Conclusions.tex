\chapter{Conclusions}

Through LEED and AES measurments a successfull preperation of both studied substrates could be validated.
Even though there could not be found a commensurate super-lattice for the 1ML coverage of ZnTPP on Cu(100), the sharp spots clearly show an ordered assembly of the molecules.
The momentum map at fermi energy as well as the main bands in the bandstructure of the Fe(100)-\textit{p}(1\times1)O sample are comparible to literature, only the features around the $\overline{\Upgamma}$ point at the fermi level could not be reproduced.
For further studies of the molecular features in k-space, it would be recommended to perform measurements at a synchrotron or collect data with higher statistics, because no molecular features where visible in the obtained momentum maps.
Nonetheless the main molecular peaks where evident in the angle integrated spectra comparable to literature. [tesi]
In addition to that, we could confirm a higher interaction for the reactive Cu(100) surface with the ZnTPP molecules in comparison to the passivated Fe(100) surface, through an observed reduced workfunction.

