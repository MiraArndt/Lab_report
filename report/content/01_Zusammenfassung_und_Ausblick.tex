\chapter{Zusammenfassung und Ausblick}

In dieser Arbeit wurde das Wachstum von MgO sowohl auf Fe(100)- als auch auf Fe(100)-p(1\,x\,1)O-Oberflächen
mit verschiedenen Methoden untersucht. 
Es wurden Daten aufgenommen, die das Wachstum charakterisieren und als mögliche Referenz dienen,
dünne Fe(100)/MgO- und Fe(100)-p(1\,x\,1)O/MgO-Schichtsysteme präzise herzustellen.
%
%Es wurden Referenzdaten dieser Methoden präsentiert, die das Wachstum charakterisieren und  
%somit eine Möglichkeit geben, dünne Fe(100)/MgO- und Fe(100)-p(1\,x\,1)O/MgO-Schichtsysteme präzise zu reproduzieren.

Die Ergebnisse zur Bestimmung der Schichtdicke mittels AES- und MEED-Messungen 
komplementieren sich gegenseitig.
Es hat sich herausgestellt, dass die ersten beiden Monolagen auf Fe(100) schneller 
wachsen und sich anschließend eine Wachstumsrate einstellt, die der konstanten Rate vom Fe(100)-p(1\,x\,1)O/MgO-System entspricht.\newline
Außerdem konnte die für solch ein System charakteristische Abweichung zwischen der ermittelten Wachstumsrate und der 
vorhergesagten QCM-Wachstumsrate bestimmt werden. 

Durch LEED-Aufnahmen vor und nach dem Erhitzen konnte ein positiver Effekt 
des Aufheizens auf die Oberflächenordnung des Fe(100)/MgO-Systems nachgewiesen werden.
Ein Aufbrechen von Inseln bei $1,6$\,ML als Ursache widerspricht den Ergebnissen von Klaua et al. \cite{klaua2001growth} und 
außerdem den Ergebnissen der durchgeführten MEED-Messungen. Als komplementäre Methode zur Untersuchung dieses Effekts 
könnte ein Rastertunnelmikroskop genutzt werden.\newline
Weiterhin wurden charakteristische Merkmale in den IV-Kurven beider Systeme 
für unterschiedliche Schichtdicken herausgearbeitet.

Die Ergebnisse für das geeignete $r/p$-Verhältnis von Tekiel et al. \cite{tekiel2013reactive} konnten nicht bestätigt werden,
denn wie die Elementanalyse mit AES gezeigt hat, entsteht beim Wachstum mehrerer Monolagen ein Sauerstoffüberschuss.\newline
Auffällig ist jedoch, dass die Schichtdickenanalyse mittels AES trotzdem passende Ergebnisse 
sowohl für das Mg/Fe- als auch für das O/Fe-Verhältnis erzeugt hat. 
Weitere Messungen bei verändertem Sauerstoffdruck wären nötig, um dieses Phänomen weiter zu untersuchen. 


Ein hier nicht betrachteter Aspekt ist der gezielte Einbau von Defekten mittels verändertem r/p-Verhältnis 
in der MgO-Schicht. So kann die Austrittsarbeit verändert werden und folglich die Eigenschaften molekularer 
Hybridsysteme gezielt manipuliert werden \cite{greiner2012universal}. 
Notwendig dazu wäre eine Vermessung der Austrittsarbeit zum Beispiel mittels Photoelektronenspektroskopie.
%Ein weiterer Aspekt des Wachstums, der hier nicht betrachtet wurde, ist der gezielte Einbau von Defekten 
%in der MgO-Schicht, um eine Änderung der Austrittsarbeit herbeizuführen.
%Solch eine Veränderung der Austrittsarbeit kann genutzt werden, um die Eigenschaften von molekularen Spinterface-Schichtsystemen 
%gezielt zu manipulieren \cite{greiner2012universal}. \newline
%Gegenstand zukünftiger Untersuchungen könnte also eine Charakterisierung des MgO-Wachstums mit verändertem 
%$r/p$-Verhältnis sein, um gezielt Defekte einzubauen. Damit verbunden wäre eine Vermessung der Austrittsarbeit 
%des Systems zum Beispiel mittels Photoelektronenspektroskopie. 


%Der Einbau solcher Defekte kann durch die hier verwendete Wachstumsmethode durch eine Veränderung des $r/p$-Verhältnises
%vorgenommen werden. Der Versuch das MgO-Wachstum bei Sauerstoffmangel zu charakterisieren ist im Rahmen dieser Arbeit fehlgeschlagen.






%Ein weiterer Effekt, der eine genauere Betrachtung wert ist, ist der Anstieg der MEED-Intensität \ref{fig:MEED1}
%direkt nach dem Öffnen des Shutters. Erwartet wird zunächst eine niedrigere Ordnung der Oberfläche durch das Auftreffen 
%der ersten Fremdatome, welche eine Senkung der Beugungsintensität zur Folge haben sollten.
%Eine solche Senkung ist auch bei den weiteren MEED-Messungen \ref{fig:MEED2} zu erkennen.
%MEED-Messungen unter verschiedenen Bedinungen, vor allem der Variation des Winkels und der vermessenen Beugungsreflexe
%könnten Aufschluss über diesen unerwarteten Effekt bringen.


