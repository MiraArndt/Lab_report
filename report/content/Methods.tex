\chapter{Methods}

The multichamber ultra high vacuum (UHV) system with a base pressure in the range of \qty{e-10}{mbar}-\qty{e-11}{mbar} used in these experiments consists of a loadlock, which can be pumped down rapidly, to get the samples inside the setup, a preparation chamber where the samples are cleaned, characterized by LEED and Auger Electron Spectroscopy (AES) and the molecule deposition takes place and lastly the analysis chamber where the measurements are taken with a momentum microscope to study the electronic structure.

The first step in the preparation of the Fe(100) surface was cleaning by multiple cycles of sputtering the sample with $\mathrm{Ag}^+$-ions at a kinetic energy of \qty{1}{keV} and an emission current of \qty{6}{mA} for 15 minutes and then annealing for 7 minutes after reaching \qty{600}{°C} to rearange the surface to be atomically flat.
Afterwards, the sample was passivated for 5 minutes at \qty{550}{°C}, exposing the surface to oxygen at a pressure of \qty{1.0e-7}{mbar}, which corresponds to about 23 Langmuirs.
Next, a flash annealing procedure upto \qty{600}{°C} took place to limit the heat treatment to the surface area and to remove the extra oxygen atoms, leaving behind a Fe(100)-\textit{p}(1\times1)O surface, where the oxygen atoms lie in the hollow sites of the Fe(100) substrate.
The Cu(100) surface underwent a similar cleaning procedure.
Here the kinetic energy for sputtering was set to \qty{2}{keV} for a duration of 70 minutes.
ZnTPP was deposited on the newly formed Fe(100)-\textit{p}(1\times1)O and the Cu(100) surfaces via Knudsen effusion.
For this the effusion cell containing the ZnTPP was heated to \qty{265}{°C} and then opened for 12 minutes aiming perpendicularly at the sample.

LEED is a basic technique to examine the crystallographic surface structure of a sample by scattering electrons elastically on the sample surface and observing the spots, generated on a flourescent screen, corresponding to the reciprocal lattice vectors due to the Laue equations.
This method is used to check the quality of the prepared substrate and to investigate the assembly of the deposited porphyrins.
To confirm the chemical composition of the prepared sample AES is used.
The electronic properties are probed via Angle Resolved Photoemission Spectroscopy (ARPES), realized with the KREIOS system, recording two dimensional momentum maps for different kinetic energies, measured by a hemispherical analyzer.
To excite the photoelectrons, high harmonic generated (HHG) light with a photon energy of \qty{29,8}{eV}, as well as a He lamp with a photon energy of \qty{21,2}{eV} was used, ideal to study the valance band near the fermi level.
For both surfaces the analyzer was set to measure higher kinetic energies to study some of the main features near the fermi level and to lower kinetic energies to measure the secondary electrons and therefore gain information about the work function, to be able to shed light on the strength of the interface dipole.
All measurements were done with an energy resolution of $\Delta E = \qty{50}{meV}$.

The data analysis was done with the software IGOR PRO 8 \cite*{igor}.
The data in the form of a stack of momentum maps for different kinetic energies was cut along the high symmetry lines of the surface brillouin zone (SBZ) to extract the bandstructure.
By integrating the intensity over all measured angles one obtains the equivalent of an Ultraviolet Photoemission Spectroscopy (UPS) spectrum.

