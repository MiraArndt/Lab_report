\chapter{Methods}

The multichamber ultra high vacuum (UHV) system with a base pressure in the range of $\qty{e-10}{mbar}$ used in these experiments consists of a loadlock to get the samples inside the setup, a preparation chamber where the samples are cleaned and the molecule deposition takes place and lastly the analysis chamber where the measurements are taken.

The first step in the preparation of the Fe(001) surface was cleaning by sputtering the sample with $\mathrm{Ag}^+$-ions at a kinetic energy of \qty{1}{keV} and an emission current of \qty{6}{mA} for 15 minutes and then annealing for 7 minutes after reaching \qty{600}{°C} to rearange the surface to be atomically flat.
Afterwards, the sample was passivated with the base pressure rising to \qty{1.0e-7}{mbar} for 5 minutes at \qty{550}{°C}, exposing the surface to about 23 Langmuirs of oxygen.
Next, a flash annealing procedure upto \qty{600}{°C} took place to limit the heat treatment to the surface area and to remove the extra oxygen atoms, leaving behind a Fe monoxide layer.
The oxygen atoms lie in the hollow sites situated on the Fe(001) surface.
The Cu(100) surface underwent a similar cleaning procedure.
Here the kinetic energy for sputtering was set to \qty{2}{keV} for a duration of 70 minutes.
Zinc tetraphenylporphyrin (ZnTPP) was deposited on the newly formed Fe(001)p-(1\times1)O and the Cu(100) surfaces via Knudsen effusion.
For this the effusion cell containing the ZnTPP was heated to \qty{265}{°C} and then opened for 12 minutes aiming perpendicularly at the sample.

Low Energy Electron Diffraction (LEED) is a basic technique to examine the crystallographic structure of a sample by scattering electrons elastically on the sample surface and observing the spots, generated on a flourescent screen, corresponding to the reciprocal lattice vectors due to the Laue equations.
This method is used to inspect the coverage of the monoxide layer and to check the quality of the super-lattice consisting of porphyrins.
To confirm the chemical composition of the prepared sample Auger Electron Spectroscopy (AES) is used.
The electronic bandstructure is probed via Ultraviolet Photoemission Spectroscopy (UPS), where the obtained spectra relay information about the valance bands near the fermi level.
The angle resolved photoemission spectroscopy (ARPES) used to record the two dimensional momentum maps is realized with the KREIOS system.
The impuls component parallel to the surface needed for the maps is computed using the emission angle of the extracted electrons, which in turn is captured by the hemispherical analyzer.
To excite the electrons, for the photoemission to happen, high harmonic generated (HHG) light with a photon energy of \qty{29,8}{eV} was used.
For both surfaces the analyzer was set to a wider range of kinetic energies to study some of the main features and to a narrow range to obtain the work function and information pertaining to the surface dipole.
All measurements were done with an energy resolution of $\Delta E = \qty{50}{meV}$.
