\chapter{Introduction}
The flexibility of porphyrins in organic devices such as solar cells \cite*{schmidt-mende_zn-porphyrin-sensitized_2005} and OLEDs \cite*{graham_extended_2011}, as well their biological importance \cite*{dolphin_biochemical_1974} are just some of the reasons why porphyrines are subject of in-depth study \cite*{gottfried_surface_2015}.
Especially the adsorbtion on metal surfaces and the placing of a metall ion in the main cavity of the molecule contribute greatly to the tunability of the porphyrin properties \cite*{calloni_electron_2014}.
The understanding of the fundamental mechanics involved in the interaction between molecule and substrate as well as the physical and chemical properties of the interface work as a basis for creating new devices.

Metall-Tetraphenylporphyrin (M-TPP) consists of a tetrapyrrolic macrocycle, which lies almost flat on the surface and therefore interacts the most with the substrate \cite*{picone_local_2018}.
The four phenyl groups are situated on the border of the macrocycle and are almost perpendicular to the macrocycle in the unperturbed state.
The metall ion located in the middle of the pyrrole ring stays in the plane of the ring, meaning that it is accessible from both sides.

As previous works show, a direct deposition of ZnTPP on a clean Fe(100) surface produced no pattern by Low Electron Energy Diffraction (LEED) \cite*{bussetti_structure_2016}, which is said to be caused by island formation disrupting the flat crystallographic structure.
Passivation of this highly reactive Fe(100) substrate was shown to prevent strong coupling to the porphyrins, so that eventhough the pyrrole ring lies flat on the substrate the main electronic features of the molecule like the HOMO and LUMO are conserved \cite*{picone_local_2018}.
After passivation of the Fe(100) substrate, the ZnTPP molecules then also form an ordered (5\times5)-reconstruction, showcasing the enhanced molecule diffusivity compared to the clean Fe(100) surface \cite*{bussetti_structure_2016}.

Here we wanted to reproduce some of the results for Fe(100)-\textit{p}(1\times1)O/ZnTPP found in literatur \cite*{bussetti_filled_2016} and build upon them to study the interface of Cu(100)/ZnTPP in a similar manner.
The unpassivated Cu surface is naturally expected to undergo greater change in its electronic structure after adsorption of the ZnTPP molecules.